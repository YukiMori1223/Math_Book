\chapter{はじめに}
\section{この文書で使用する数学}\label{sec:math_prepare}
ここでは、本文書で使用している種々の数学的知識について、高校数学までを前提としながら、大学数学を振り返る。
この章においては必要な公式を振り返りやすくするために定義および命題\footnote{定義から導かれる事実を、数学では定理、命題、補題、系などと呼ぶ。これらは重要性によって呼び分けられており、最も重要なものを「定理」、それに準じるものを「命題」などと呼ぶようである。本章は基本的な事実ばかりを述べており、「定理」と呼ぶほどではないと考えたため、基本的に「命題」に統一している。}を囲んだ。また、証明はできる限り記述したが、より一般的な証明や拡張についてはそれぞれの分野の教科書を参照されたい。

大学での数学講義は集合論、線形代数、および微分積分学の3分野から始まる場合が多い。これらの分野は本文書の内容に対しても非常に重要な基礎であるが、行列の基本変形など本文書の範囲では不要な部分も存在する。ここでは、数学の基礎部分である集合論、線形代数について必要な部分に絞って述べる。次に、本文書で多用するテンソル、ベクトル解析、およびクォータニオンについて述べる。テンソルは線形代数を拡張した概念で、特に材料力学や流体力学などの連続体力学に多く現れる。ベクトル解析はベクトル場の微分積分などを扱う分野で、空間上の流れを扱う流体力学では非常に重要である。クォータニオンは、数学的には複素数を拡張した数の一種であるが、本文書では主に固体粒子や物体の姿勢を示す値として扱われる。

\section{アインシュタインの縮約記法(総和規約)}
本文書では、いわゆる\emph{アインシュタインの縮約記法}\index{あいんしゅたいん@アインシュタインの縮約記法}と呼ばれる記法を用いていない。これは、以下のようなルールで記述を行う方法である。
\begin{definition*}{アインシュタインの総和記法(本文書では使用しない)}
	添字\(i,j,k,...\)、と、それが取りうる文字\(x,y,z,...\)に対し、以下のような規約を設ける。
	\begin{itemize}
		\item{同じ項に同じ添字が1度現れる場合には、通常の意味である(和を取らない)。}
		\item{同じ項に同じ添字が2度現れる場合には、それが取りうる値全ての和をとる。}
		\item{同じ項に同じ添字が3度以上現れてはいけない。}
	\end{itemize}
\end{definition*}
この中で重要なのは、2度現れる文字に対して和を取るという部分であり、この規則によって総和記号\(\sum\)の記述を削減できる。この記法を用いれば、例えば3次元でのベクトルの内積は以下の数式における最右辺のような形で記述できる。
\begin{equation}
	\boldsymbol{v}\cdot\boldsymbol{w}= v_x w_x +v_y w_y+ v_z w_z=\sum_{i=x,y,z}^3{v_{i}w_{i}}={v_{i}w_{i}}
\end{equation}
この記法は、特にテンソル演算を多用する連続体力学や量子力学などの分野では非常に有効な記法である。もちろん、流体力学も連続体力学の範疇であるため、このような記法を用いて記述している文書は非常に多い。しかし、剛体の力学においてはほとんど使用しないという点、および同じ添字であっても2度登場するか否かで意味が変わるため、初学者にとっては読みにくいと考えられるという点から、本文書ではこの記法を採用していない。項を足し合わせる際には、\(\Sigma\)を明記するか、または全ての項を書き下している。
